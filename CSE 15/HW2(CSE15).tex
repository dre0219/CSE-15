\documentclass[11pt]{article}

% To produce a letter size output. Otherwise will be A4 size.
\usepackage[letterpaper]{geometry}

% For enumerated lists using letters: a. b. etc.
\usepackage{enumitem}

\topmargin -.5in
\textheight 9in
\oddsidemargin -.25in
\evensidemargin -.25in
\textwidth 7in

\begin{document}

% Edit the following putting your first and last names and your lab section.
\author{Andre Martin\\
Lab CSE-015-07L F 1:30-4:20PM}

% Edit the following replacing X with the HW number.
\title{CSE 015: Discrete Mathematics\\
Fall 2019\\
Homework \#2\\
Solution}

% Put today's date in the following.
\date{October 11, 2019}
\maketitle

% ========== Begin questions here
\begin{enumerate}

\item
\textbf{Question 1:}
\begin{enumerate}[label=(\alph*)]

\item{Every rabbit hops}

\item{All animals are rabbits and they hop}

\item{There is an animal that if it is a rabbit, then it hops}

\item{There is an animal that is a rabbit and it hops}


\end{enumerate}

\item
\textbf{Question 2:}

\begin{enumerate}[label=(\alph*)]
\item{True}
\item{False}
\item{True}
\item{False}
\end{enumerate}

\item
\textbf{Question 3:}
\begin{enumerate}[label=(\alph*)]

\item{Let P(x) be "x has cellphone". Domain is all your students. First case logical expressions is $\forall$xP(x). Let C(x) be the statement "x in your class" and the logical expression is $\forall$x(C(x) $\wedge$ P(x)).}

\item{Let F(x) be "x has seen a foreign movie." Domain is all your students. Logical Expression is $\exists$xF(x). Let C(x) be the statement "x is in your class" and the logical expression is $\exists$x(C(x) $\wedge$ F(x)).}

\item{Let S(x) be "x can swim." Domain is all your students. Logical expression is $\exists \neg S(x)$. Let C(x) be the statement "x is in your class" and the logical expression is $\exists$x(C(x) $\wedge$ \neg S(x)).}

\item{Let Q(x) be "x can solve quadratic equations." Domain is all your students. Logical expression is $\forall$xQ(x). Let C(x) be the statement "x is in your class" and the logical expression is $\forall$x(C(x) $\wedge$ Q(x)).}

\end{enumerate}

\item
\textbf{Question 4:}
\begin{enumerate}[label=(\alph*)]

\item{There exists a student in your class that has taken at least one computer science course at your school.}

\item{Every student in your class has taken at least one computer science course}

\item{There exists a student in your class that has taken all of the computer science courses at your school}

\end {enumerate}

\item
\textbf{Question 5:}
\begin{enumerate}[label=(\alph*)]

\item{$\exists$x$\exists$y Q(x,y)}

\item{$\forall$x$\forall$y\neg Q(x,y).}

\end{enumerate}

\item
\textbf{Question 6:}
\begin{enumerate}[label=(\alph*)]

\item{$\forall$x$\exists$yF(x,y)}

\item{$\neg$$\exists$x$\forall$yF(x,y)}

\item{$\exists$y$\forall$xF(x,y)}

\end{enumerate}

\item
\textbf{Question 7:}
\begin{enumerate}[label=(\alph*)]

\item{$\forall$x$\forall$y(((x $<$ 0) $\wedge$ (y $<$ 0)) \rightarrow (((x*y > 0)))) }

\item{$\forall$x$\forall$y(((x $>$ 0) $\wedge$ (y $>$ 0)) \rightarrow (((x+y/2 > 0)))).}

\item{$\forall$x$\forall$y((\left| (x+y) \right| < (\left| x \right| + \left| y \right|)))}

\end{enumerate}

\item
\textbf{Question 8:}
\begin{enumerate}[label=(\alph*)]

\item{False}

\item{True}

\item{False}


\end{enumerate}

\end{enumerate}

\end{document}
